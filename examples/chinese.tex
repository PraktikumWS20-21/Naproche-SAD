\documentclass[a4paper,draft]{amsproc}
\title{\textbf{Chinese}}

\date{}
\begin{document}

\theoremstyle{plain}
 \newtheorem{theorem}{Theorem}[section]
 \newtheorem{lemma}{Theorem}[section]
 \newtheorem{proposition}{Theorem}[section]
\theoremstyle{definition}
 \newtheorem{example}{Example}[section]
 \newtheorem{definition}{Definition}[section]
 \newtheorem{signature}{Signature}[section]
\theoremstyle{remark}
 \newtheorem{remark}{Remark}[section]
 \newtheorem{notation}{Notation}[section]
\theoremstyle{axiom}
 \newtheorem{axiom}{Axiom}[section]
 \numberwithin{equation}{section}

\newenvironment{forthel}{}{}
\maketitle

This file only contains the beginning of 'chinese.ftl', but in tex format. The point of this file is testing whether labels given to environments are properly parsed.

\begin{forthel}
[synonym element/-s]

\begin{signature}[ElmSort]
An element is a notion.
\end{signature}

Let a,b,c,x,y,z,u,v,w denote elements.

\begin{signature}[SortsC]
0 is an element.
\end{signature}

\begin{signature}[SortsC]
1 is an element.
\end{signature}

\begin{signature}[Sortsu]
-x is an element.
\end{signature}

\begin{signature}[SortsB]
x + y is an element.
\end{signature}

\begin{signature}[SortsB]
x * y is an element.
\end{signature}

\begin{axiom}[AddComm]
x + y = y + x.
\end{axiom}

\begin{axiom}[AddAsso]
(x + y) + z = x + (y + z).
\end{axiom}

\begin{axiom}[AddBubble]
x + (y + z) = y + (x + z).
\end{axiom}

\begin{axiom}[AddZero]
x + 0 = x = 0 + x.
\end{axiom}

\begin{axiom}[AddInvr]
x + -x = 0 = -x + x.
\end{axiom}

\begin{axiom}[MulComm]
x * y = y * x.
\end{axiom}

\begin{axiom}[MulAsso]
(x * y) * z = x * (y * z).
\end{axiom}

\begin{axiom}[MulBubble]
x * (y * z) = y * (x * z).
\end{axiom}

\begin{axiom}[MulUnit]
x * 1 = x = 1 * x.
\end{axiom}

\begin{axiom}[AMDistr1]
x * (y + z) = (x * y) + (x * z).
\end{axiom}

\begin{axiom}[AMDistr2]
(y + z) * x = (y * x) + (z * x).
\end{axiom}

\begin{axiom}[MulMnOne]
1 * x = -x = x * -1.
\end{axiom}

\begin{lemma}[MulZero]
x * 0 = 0 = 0 * x.
Proof.
    Let us show that x * 0 = 0.
    x * 0 .= x * (0 + 0) (by AddZero) .= (x * 0) + (x * 0) (by AMDistr1).
    end.
    Let us show that 0 * x = 0.
    0 * x .= (0 + 0) * x (by AddZero) .= (0 * x) + (0 * x) (by AMDistr2).
    end.
qed.
\end{lemma}
\end{forthel}

\end{document}